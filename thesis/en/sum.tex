\chapter{Conclusion}
\label{ch:sum}

Quantum circuit design tools often force users to choose between visual simplicity and computational power. Qubit bridges this gap by connecting an interactive web-based circuit composer directly to SQUANDER's advanced decomposition algorithms. This integration proves accessibility and capability can coexist.

The platform's architecture offers insights beyond quantum computing. WebSocket-based progress streaming transforms batch computations into interactive experiences. SSH connection pooling enables seamless access to remote servers without exposing infrastructure complexity. Temporal state management with Zundo provides intuitive undo/redo functionality for complex circuit editing. These patterns apply broadly to scientific web applications that require high-performance computational backends.

Developing Qubit provided hands-on experience with distributed systems, asynchronous communication, and full-stack architecture for scientific computing. The project demonstrated how modern web technologies can make specialized computational tools accessible to broader audiences while maintaining the depth required for research and education.

As quantum computing moves from theory to practice, the ecosystem needs more tools that lower barriers without limiting capability. Qubit represents one step in this direction, enabling researchers and students to explore quantum algorithms through visual interfaces while leveraging powerful decomposition engines running on remote servers.