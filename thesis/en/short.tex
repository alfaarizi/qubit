\section{Shortcomings and Limitations}
\label{sec:shortcomings}

Qubit demonstrates that web-based platforms can effectively bridge visual circuit design with advanced simulation capabilities. However, several architectural constraints limit the system's scope and scalability. This section documents the most significant limitations, understanding which provides context for the system's current design and suggests directions for future work.

\subsection{Scalability Constraints}
\label{subsec:shortcomings:scalability}

The SSH connection pool limits concurrent simulations to five active sessions. While this constraint prevents resource exhaustion on SQUANDER servers, it creates a throughput bottleneck when multiple users submit jobs simultaneously. The connection pool uses a simple FIFO queue without priority handling, so quick validation runs must wait behind lengthy optimization tasks. This is particularly constraining for classroom scenarios where dozens of students run circuits at the same time.

The WebSocket architecture maintains all active connections and job state in memory within a single backend process. This prevents horizontal scaling across multiple servers without implementing distributed state management through Redis or similar infrastructure. As the number of concurrent users grows, a single instance eventually exhausts available memory. Production deployments would require either connection limits, aggressive state cleanup, or a Redis-based architecture to enable stateless server instances that coordinate through centralized message brokers.

\subsection{Remote Execution Dependency}
\label{subsec:shortcomings:execution}

Qubit cannot simulate quantum circuits locally—all computation requires SSH access to external SQUANDER servers. This fundamental dependency prevents deployment for local educational use or in environments without external network access. While delegating computation to remote servers enables advanced simulation capabilities that browsers cannot provide, it also means the platform is useless without this infrastructure. Local simulation support, even if limited to small circuits, would increase flexibility for offline exploration and educational deployment.

The authentication system similarly requires external dependencies: Google OAuth, Microsoft Azure AD, and email verification all require internet connectivity to identity providers. The platform provides no offline authentication mode or local credential storage, limiting deployment in environments without internet access or at institutions that restrict external connections.

\subsection{Circuit Size and Visualization Limits}
\label{subsec:shortcomings:visualization}

The D3.js circuit renderer efficiently displays circuits with 2000 or more gates through batch rendering optimizations. However, interactive DAG operations like dragging gates become slow in large circuits. While the dependency recalculation is efficient---updating only affected descendant gates---the rendering becomes the bottleneck. When users move a gate, the entire circuit state updates, triggering D3 to re-render all gates even though only a small subset changed. Moving a single gate in a 2000-gate circuit can cause UI lag lasting several seconds. The current implementation lacks targeted rendering optimizations such as updating only the affected visual elements rather than re-drawing the complete circuit diagram.

\subsection{Gate Library and QASM Compatibility}
\label{subsec:shortcomings:gates}

The gate library supports 14 common quantum operations but excludes advanced features used in production quantum algorithms. The platform cannot construct controlled versions of arbitrary gates (e.g., controlled-U gates), limiting implementations of algorithms like Quantum Phase Estimation. Additionally, the QASM parser implements only OpenQASM 2.0, excluding 3.0 features like classical control flow and mid-circuit measurements. This reduces interoperability with modern quantum frameworks and prevents expressing certain algorithmic patterns without circuit unrolling.

Measurement operations are restricted to computational basis measurements, preventing implementation of algorithms requiring measurement in alternative bases or mid-circuit measurements with classical feedback. This limits exploration of quantum error correction and measurement-based protocols.
